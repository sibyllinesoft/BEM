\subsection{Failure Mode Analysis}

To understand the mechanisms behind BEM's robustness advantage, we conducted detailed failure mode analysis examining specific cases where Static LoRA degrades significantly while BEM maintains performance.

\subsubsection{Case Study 1: Medical-to-Legal Domain Shift}

\textbf{Example Query:} "What is the statute of limitations for breach of contract claims?"

\textbf{Retrieved Context:} Legal documents discussing contract law and limitation periods.

\textbf{Analysis:}
\begin{itemize}
    \item \textbf{Static LoRA Response:} "The therapeutic window for this treatment protocol typically extends 24-48 hours..." (Incorrectly applying medical terminology patterns)
    \item \textbf{BEM P3 Response:} "The statute of limitations for breach of contract claims is typically 3-6 years, depending on jurisdiction..." (Correctly adapts to legal domain)
\end{itemize}

\textbf{Root Cause:} Static LoRA's fixed parameters encode medical domain priors (terminology, response patterns) that actively interfere with legal reasoning. BEM's dynamic parameter generation recognizes the legal context from retrieved passages and generates appropriate domain-specific adaptations.

\subsubsection{Case Study 2: Low-Quality Retrieval Scenario}

\textbf{Query:} "How does photosynthesis work?"

\textbf{Retrieved Context:} Mix of relevant biology passages and irrelevant documents about photography equipment.

\textbf{Performance Impact:}
\begin{itemize}
    \item \textbf{Baseline:} 0.425 (degraded due to noise)
    \item \textbf{Static LoRA:} 0.397 (amplifies noise through fixed adaptation)
    \item \textbf{BEM P3:} 0.419 (robust to retrieval quality)
\end{itemize}

\textbf{Mechanism Analysis:} Static LoRA applies uniform adaptation regardless of passage relevance, amplifying both signal and noise. BEM's parameter generation includes implicit quality assessment - it generates weaker adaptation parameters when evidence is mixed or contradictory.

\subsubsection{Adaptation Parameter Dynamics}

We analyzed the parameter norms generated by BEM P3 across different scenarios:

\begin{table}[h]
\centering
\caption{Dynamic Parameter Characteristics by Scenario}
\label{tab:param_dynamics}
\small
\begin{tabular}{l|c|c|c}
\toprule
\textbf{Scenario} & \textbf{Avg Parameter Norm} & \textbf{Norm Std Dev} & \textbf{Adaptation Strength} \\
\midrule
Baseline (Clean) & 0.85 & 0.12 & High \\
Distribution Shift & 0.42 & 0.18 & Reduced \\
Low-Quality Retrieval & 0.31 & 0.23 & Minimal \\
Multi-Task Interference & 0.58 & 0.15 & Moderate \\
Out-of-Distribution & 0.39 & 0.19 & Reduced \\
\bottomrule
\end{tabular}
\end{table}

This analysis reveals BEM's adaptive mechanism: parameter norms automatically decrease in challenging scenarios, providing natural regularization that prevents overfitting to poor or mismatched evidence.

\subsubsection{Computational Overhead Analysis}

While BEM provides robustness advantages, it incurs additional computational cost:

\begin{table}[h]
\centering
\caption{Computational Efficiency Comparison}
\label{tab:efficiency}
\small
\begin{tabular}{l|c|c|c}
\toprule
\textbf{Method} & \textbf{Inference Time (ms)} & \textbf{Memory (GB)} & \textbf{Robustness Score} \\
\midrule
Baseline & 145 & 3.2 & 0.97 \\
Static LoRA & 152 (+4.8\%) & 3.4 & 0.82 \\
BEM P3 & 189 (+30.3\%) & 3.8 & 0.96 \\
\bottomrule
\end{tabular}
\end{table}

BEM requires 30\% additional compute for parameter generation, but this overhead is often justified by the robustness benefits in production environments where adaptation quality is critical.

\subsubsection{Deployment Decision Framework}

Based on this analysis, we propose a decision framework for choosing adaptation methods:

\begin{itemize}
    \item \textbf{Use Static LoRA when:} Domain is stable, retrieval quality is high, computational efficiency is critical
    \item \textbf{Use BEM when:} Domain shift expected, retrieval quality variable, robustness prioritized over efficiency
    \item \textbf{Hybrid Approach:} Use BEM uncertainty estimates to selectively apply dynamic adaptation only when needed
\end{itemize}